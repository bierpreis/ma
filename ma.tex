\documentclass[12pt,a4paper]{article}
\usepackage[utf8]{inputenc}
\usepackage[german]{babel}
\usepackage[T1]{fontenc}
\usepackage{amsmath}
\usepackage{amsfonts}
\usepackage{amssymb}
\usepackage{graphicx}
\usepackage[left=2cm,right=2cm,top=2cm,bottom=2cm]{geometry}
\author{Frank Leuchtmann}
\title{Systematische Generierung konsistenter Mengen qualitativer
Konditionale}



\begin{document}
\maketitle
\newpage
\tableofcontents
\newpage
\section{Einleitung}
Diese Arbeit hat zum Ziel, Wissensbasen aus qualitativen Konditionalen auf eine systematische Art und Weise zu generieren. Dabei soll ergründet werden, welchen Aufwand diese Generierung mit steiger Anzahl der Konditionalen verursacht und wo die vertretbaren Grenzen dessen liegen.
\\
Der weitere Ablauf der Arbeit ist wie folgt: Im nächsten Kapitel werden zuerst die Grundlagen zu Konditionalen und Wissensbasen dargestellt um die Basis für das weitere vorgehen zu schaffen. Anschließend werden die Konditionale eine Normalform gebracht, vollständig berechnet und geordnet gespeichert.
\section{Grundlagen}
Ein Konditional ist die Verbindung aus einer Vorbedingung und einer Folgerung aus der Vorbedingung. Ein qualitatives Konditional ist ein solches unter  Unsicherheit und entspricht einem Zusammenhang in der Form \grqq Wenn $A$, dann normalerweise $B$\grqq .
Diese Arbeit beschränkt sich dabei auf Konditionale, die über der Signatur $\sum_{abc} = \{a, b, c\}$ gebildet werden können.
\cite{beierle17}
\section{Generierung der Konditionale in Normalform}
\section{Generierung der Wissensbasen}

\bibliographystyle{plain}
\bibliography{lit} 
\end{document}