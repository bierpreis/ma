\documentclass[12pt,a4paper]{article}
\usepackage[utf8]{inputenc}
\usepackage[german]{babel}
\usepackage[T1]{fontenc}
\usepackage{amsmath}
\usepackage{amsfonts}
\usepackage{amssymb}
\usepackage{graphicx}
\usepackage[left=2cm,right=2cm,top=2cm,bottom=2cm]{geometry}
\author{Frank Leuchtmann}
\title{Systematische Generierung konsistenter Mengen qualitativer
Konditionale}



\begin{document}
\maketitle
\newpage
\tableofcontents
\newpage
\section{Einleitung}
In dieser Arbeit geht es darum
\\
Der weitere Ablauf der Arbeit ist wie folgt: Im nächsten Kapitel werden zuerst die Grundlagen zu Konditionalen und Wissensbasen dargestellt um die Basis für das weitere vorgehen zu schaffen. Anschließend werden die Konditionale in Normalform in den Mengen $NFC\sum_{abc}$ und $cNFC\sum_{abc}$ berechnet.
\section{Grundlagen}
\cite{beierle17}
\section{Generierung der Konditionale in Normalform}
\section{Generierung der Wissensbasen}

\bibliographystyle{plain}
\bibliography{lit} 
\end{document}