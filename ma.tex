\documentclass[12pt,a4paper]{article}
\usepackage[utf8]{inputenc}
\usepackage[german]{babel}
\usepackage[T1]{fontenc}
\usepackage{amsmath}
\usepackage{amsfonts}
\usepackage{amssymb}
\usepackage{graphicx}
\usepackage[left=2cm,right=2cm,top=2cm,bottom=2cm]{geometry}
\newtheorem{theorem}{Definition}
\author{Frank Leuchtmann}
\title{Systematische Generierung konsistenter Mengen qualitativer
Konditionale}

% this value solves the problem overfull hbox. it defines somehow max empty space in a line.
\tolerance=600

\newcommand{\lag}{\mathcal{L}}

%%the following 2 is to define the dotl command. can this be shorter??
\newcommand\dotl{\mathrel{%
    \mathchoice{\QEQ}{\QEQ}{\QEQ}{\QEQ}%
}}
\def\QEQ{{%
    \setbox0\hbox{<}%
    \rlap{\hbox to \wd0{\hss \raisebox {0.6pt}{$\ \cdot$\hss}}}\box0
}}

\newcommand\dotll{\mathrel{%
    \mathchoice{\QEQQ}{\QEQQ}{\QEQQ}{\QEQQ}%
}}
\def\QEQQ{{%
    \setbox0\hbox{$\leqslant$}%
    \rlap{\hbox to \wd0{\hss \raisebox {1pt}{$\ \cdot$\hss}}}\box0
}}

\begin{document}
\maketitle
\newpage
\tableofcontents
\newpage
\section{Einleitung}
Diese Arbeit hat zum Ziel, Wissensbasen aus qualitativen Konditionalen auf eine systematische Art und Weise zu generieren. Dafür wird der in \cite{beierle19} entwickelte Algoritmus zur Berechnung aller Wissensbasen über der Signatur $\sum_{abc}$ implementiert. Dabei soll ergründet werden, welchen Aufwand diese Generierung mit einer steigender Anzahl der darin  enthaltenden Konditionalen verursacht. Die als Ergebnis entstandenen Wissensbasen sollen dann dazu dienen, das System InfOCF (siehe \cite{beierle17}) experimentell zu evaluieren.
\\
Der weitere Ablauf dieser Arbeit ist wie folgt: Im nächsten Kapitel werden zuerst die Grundlagen zu Konditionalen und Wissensbasen dargestellt um die Basis für das weitere vorgehen zu schaffen. Anschließend werden die betrachteten Koditionale in einer Normalform vollständig berechnet, dargestellt und geordnet gespeichert. Im Hauptteil dieser Arbeit wird dann ein Algoritmus aus \cite{beierle19} zur Erzeugung aller Wissensbasen über der Signatur implementiert und die Vorgehensweise dabei beschrieben. Die erhaltenen Ergebnisse werden dann in einer geeigneten Form dargestellt. Abschließend werden die gewonnenen Ergebnisse bewertet.
\section{Grundlagen}
In diesem Abschnitt werden kurz die Grundlagen aufgeführt, auf denen der Hauptteil dieser Arbeit dann basiert. Die Inhalte und Darstellungen sind dabei aus den Kapiteln über Grundlagen aus \cite{beierle19} und \cite{beierle17} übernommen.
\subsection{Konditionale}
Zuerst wird kurz die Notation beschrieben, mit der die Konditionale im weiteren Verlauf der Arbeit dargestellt werden. Die einzelnen Atome, die betrachtet und beschrieben werden, erhalten Bezeichnungen mit Kleinbuchstaben $a, b, c,...\ $. Aussagen über diese werden in Form von Großbuchstaben $A, B, C,...$ formuliert, die Gesamtheit der Sprache wird als $\lag$ bezeichnet. Die Menge der möglichen Welten, die betrachtet werden, wird mit $\Omega$ bezeichnet. In einer dieser Welten $\omega \in \Omega$  hat der Ausdruck $\omega \models A$ die Bedeutung, dass $A \in \lag$ in dieser Welt gilt. Des weiteren wird $\neg A$ als $\overline{A}$ und $A \wedge B$ als $AB$ dargestellt.\\
Ein Konditional ist die Verbindung aus einer Vorbedingung und einer Folgerung aus der Vorbedingung. Ein qualitatives Konditional ist ein solches unter  Unsicherheit und entspricht einem Zusammenhang in der Form \glqq Wenn $A$, dann normalerweise $B$\grqq . Dargestellt wird so ein Konditional  im folgenden mit dem Symbol \glqq$|$\grqq \space in der Form $r = ( \lag | \lag)$. Ein Beispiel ist $r = (B|A)$, das dazugehörige gegenteilige Konditional lautet dann $\overline{r} = (\overline{B}|A)$.\\
Das Konditional unter Unsicherheit in der Form $B|A$ teilt dann die Menge möglicher Welten $\Omega$ in drei Teile auf: Welten, die das Konditional bestätigen, Welten die es falsifizieren und Welten, auf die das Konditional nicht zutrifft weil die Vorbedingung nicht erfüllt ist. Die letztere Möglichkeit wird mit $u$ bezeichnet, was etwa als \textit{unknown} gedeutet werden kann.

\[
  (B|A)(\omega)=\begin{cases}
               1 \quad wenn \quad \omega \quad \models AB\\
               0 \quad wenn \quad \omega \quad \models A\overline{B}\\
               u \quad wenn \quad \omega \quad \models \overline{A}
            \end{cases}
\]

Ein Konditional wird als wahr akzeptiert, wenn es plausibler ist als das zugehörige kontrafaktische Konditional. Beispielsweise $(B|A)$ wird dann akzeptiert, wenn es plausibler ist als $\overline{B}|A$.
\subsection{Äquivalente und triviale Konditionale}
Äquivalenz für Konditionale soll im weiteren Verlauf dadurch charakterisiert sein, dass zwei äquivalente Konditionale die Menge der möglichen Welten auf die gleiche Art und Weise unterteilen:
\begin{equation}
(B|A)\equiv (B^\prime|A^\prime) \quad gdw. \quad A\equiv A^\prime \quad and \quad AB \equiv A^\prime B^\prime
\end{equation}
Als triviale Konditionale werden solche beschrieben, die keinen Mehrwert bieten. Das beinhaltet sowohl sich selbst erfüllende in der Form $(A \models A)$ als auch Konditionale, die ein anderes einfach umkehren, in Form von $(A \models \overline{B})$. Solche Konditionale sind bei der weiteren Betrachtung nicht von Interesse und werden daher nicht weiter beachtet.
\subsection{Ordnungsrelation für Konditionale}
In diesem Abschnitt wird eine Ordnungsrelation für Konditionale definiert, unter der Konditionale später sowohl geordnet bearbeitet als auch gespeichert werden können. Als Hilfsmittel dazu wird $ <_{lex} $  verwendet, was die alphabetische Ordnung über die vorkommenden Buchstaben beschreiben soll, beispielsweise $a <_{lex} b <_{lex} c ...$ .
\begin{theorem}[Ordnungsrelation für Mengen S]\ \\
Für die geordneten Mengen $S, S^\prime$ mit $S = \{e_1, ..., e_n\}$ und $S^\prime = \{e_{1}, ... , e_{n\prime}\}$ gilt:
\begin{equation}
 S \leqslant_{set} S ^\prime \quad gdw. \quad n<n^\prime, \text{ oder } n = n ^\prime \text{ und } e_1...e_n  \leqslant_{lex}  e_1^\prime ... e_{n^\prime}^\prime
\end{equation}
\end{theorem}
Bei der Signatur $\sum = \{ a, b, c\}$ mit der Ordnungsrelation $a \dotl b \dotl c$ ergibt sich eine Ordnungsrelation für die möglichen Mengen wie folgt: $\{a\} \dotll_{set} \{b\} \dotll_{set} \{c\} \dotll_{set} \{a,b\} \dotll_{set} \{a,c\} \dotll_{set}\{b,c\} \dotll_{set} \{a, b, c\}$. \\
Im weiteren Verlauf wird für die Darstellung von Konditionalen auf die Notation mittels Mengen möglicher Welten verwendet in Form von $\Omega_F = \{\omega | \omega \models F \}$, beispielsweise: \\
$\Omega_{a\vee\overline{b}\vee \overline{c}} = \{abc,ab\overline{c},a\overline{b}c,a\overline{b}\overline{c},\overline{a}b\overline{c},\overline{a}\overline{b}c,\overline{a}\overline{b}\overline{c}\}$ \\
Um diese Notation etwas kompakter darzustellen, wird eine Darstellung $[[\omega]]_{\dotl}$ für eine mögliche Welt unter der Signatur $\sum{abc}$ per Zahl definiert, angelehnt an deren Interpretation als Binärzahl: \\
$[[abc]]_{\dotl}  = 7,\ 
[[ab\overline{c}]]_{\dotl} = 6,\ 
[[a\overline{b}c]]_{\dotl} = 5,\ 
[[a\overline{b}\overline{c}]]_{\dotl} = 4 ,\
[[\overline{a}bc]]_{\dotl} = 3,\ 
[[\overline{a}b\overline{c}]]_{\dotl} = 2 ,\  
[[\overline{a}\overline{b}c]]_{\dotl}  = 1,\ 
[[\overline{a}\overline{b}\overline{c}]]_{\dotl} = 0 $ \\
Mit dieser Notation sieht das Beispiel von oben wie folgt aus: \\
$\Omega_{a\vee\overline{b}\vee \overline{c}} = \{ 7,6,5,4,2,1,0\}$
\begin{theorem}[Ordnungsrelation für Welten und Konditionale]
Mit der gerade beschriebenen Ordnung lässt sich nun eine Ordnungsrelation $\overset{\mathrm{\omega}}{\dotll}$ für Welten $\omega, \omega'$ und $\overset{\mathrm{c}}{\dotll}$ Konditionale $(B|A), (B',A')$ definieren:
\begin{equation}
\omega \overset{\mathrm{\omega}}{\dotll} \omega' \quad gdw. \quad [[\omega]]_{\dotl} \geqslant [[\omega']]_{\dotl}
\end{equation}

\begin{equation}
(B|A) \overset{\mathrm{c}}{\dotll} (B',A') \quad gdw. \quad \Omega_A \overset{\mathrm{\omega}}{\dotll}_{set} \Omega_{A'} \quad oder \quad \Omega_A =  \Omega_{A'} \quad und \quad  \Omega_B \overset{\mathrm{\omega}}{\dotll}_{set} \Omega_{B'}
\end{equation}

\end{theorem}
Da sich die Unterscheidung von $\overset{\mathrm{\omega}}{\dotll}$ und $\overset{\mathrm{c}}{\dotll}$ aus dem Kontext ergibt, wird der Einfachheit halber im weiterem Verlauf $\dotl$ verwendet. \\
Mit der Signatur $\sum_{abc}$ ergibt sich für mögliche Welten beispielsweise $abc \dotl ab\overline{c} \dotl \overline{a}bc \dotl \overline{a} \overline{b} \overline{c}$. Ein Beispiel für die Ordnungsrelation für Konditionale ist $(abc|abc \vee ab \overline{c}) \dotl (abc | abc \vee \overline{a} \overline{b} \overline{c} )$ oder $(abc \vee \overline{a} \overline{b} \overline{c} | abc \vee ab \overline{c} \vee \overline{a} \overline{b} \overline{c}) \dotl (\overline{a} \overline{b} \overline{c} |  abc \vee ab \overline{c} \vee \overline{a} bc \vee \overline{a} \overline{b} \overline{c})$.
\subsection{Normalform für Konditionale}
Im folgenden wird eine Normalform für Konditionale beschrieben. Das Ziel ist die klare Definition einer vollständigen und minimalen  Menge an nichttrivialen Konditionalen über einer Signatur $\sum$, die untereinander nicht äquivalent sind. Diese Normalform wird mit $NFC(\sum)$ bezeichnet.

\begin{theorem}
Die folgende charakterisierung beschreibt die Menge der Konditionale in Normalform über einer gegebenen Signatur: \\
$NFC(\sum) = \{(B|A)|A \subseteq \Omega_A, B \subsetneq A, B \neq \emptyset \}$ \\
Für diese Menge gelten dann die folgenden drei Eigenschaften:\
 \begin{itemize}
\item{Nichttrivialität: $NFC$ enthält kein triviales Konditional.}
\item{Komplettheit: für jedes nichttriviale Konditional in $\sum$ existiert ein äquivalentes Konditional in $NFC$ }
\item{Minimalität: alle Konditionale in $NFC(\sum)$ sind paarweise nicht äquivalent.}
\end{itemize}
\end{theorem}

\subsection{Wissensbasen}
Eine endliche Menge $\mathcal{R} \subseteq (\lag | \lag)$ an Konditionalen wird als Wissensbasis bezeichnet. Um die Konsistenz einer Wissensbasis zu klären, werden ordinale Rangfunktionen verwendet. Eine Wissensbasis wird dann als konsistent bezeichnet, wenn eine konditionale Rangfunktion existiert, die alle Konditionale in $\mathcal{R}$ akzeptiert.\\
Rangfunktionen wurden von Wolfgang Spohn entwickelt, siehe dazu auch \cite{spohn12}. Eine Rangfunktion ist eine Funktion in der Form $\kappa :  \Omega \rightarrow \mathbb{N} $. Um mehrere mögliche Welten zu beschreiben, wird eine Menge von Rangfunktionen erstellt. Mit einer Rangfuntion wird einer Welt $\kappa$ aus den möglichen Welten $\Omega$ eine natürliche Zahl zugeordnet, um deren Grad an Plausibilität zu beschreiben.  Die normalste aller Welten bekommt den Wert 0, weniger plausiblere Welten bekommen aufsteigend entsprechend ihrer Unplausibilität höhere Werte zugewiesen. Ein Konditional wird in diesem System akzeptiert, wenn es plausibler ist und damit einen geringeren Rang besitzt, als dessen Negation, z.B. $\kappa(AB)<\kappa(A\overline{B})$. \\
wann wird eine rangfunktion akzeptiert? \\
hier noch was daraus? : \cite{beierle2017b} \\
äquivalenz und isomorphismen aus orginalquelle
\section{Berechnung der Konditionale in Normalform}
In diesem Abschnitt werden die Mengen der Konditionale $NFC$ und $cNFC$ berechnet und aufgelistet. Diese entstandene Mengen bilden die Basis um die Wissensbasen im nächsten Abschnitt zu berechnen. \\
!!! wie viele Konditionale wären theoretisch möglich?  siehe in orginalquelle auf seite 3 mitte. wie sieht das für abc aus???\\
zur menge der konditionale: was ist der betrag, die kardinalität der signatur? hat das was mit der schreibweise, also {1,2,3} vs $\Omega_{a \vee \overline{b}}$ zu tun?
\section{Generierung der Wissensbasen}
\subsection{Vorüberlegungen}
\subsection{Implementierung des Algorithmus}
\subsection{Generierung der Wissensbasen}
\subsection{Darstellung des Ergebnisses}
\section{Bewertung der Ergebnisse}

\newpage
\bibliographystyle{alpha}
\bibliography{lit} 
\end{document}