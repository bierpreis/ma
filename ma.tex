\documentclass[12pt,a4paper]{article}
\usepackage[utf8]{inputenc}
\usepackage[german]{babel}
\usepackage[T1]{fontenc}
\usepackage{amsmath}
\usepackage{amsfonts}
\usepackage{amssymb}
\usepackage{graphicx}
\usepackage[left=2cm,right=2cm,top=2cm,bottom=2cm]{geometry}
\newtheorem{theorem}{Definition}
\author{Frank Leuchtmann}
\title{Systematische Generierung konsistenter Mengen qualitativer
Konditionale}

\newcommand{\lag}{\mathcal{L}}



\begin{document}
\maketitle
\newpage
\tableofcontents
\newpage
\section{Einleitung}
Diese Arbeit hat zum Ziel, Wissensbasen aus qualitativen Konditionalen auf eine systematische Art und Weise zu generieren. Dafür wird der in \cite{beierle19} entwickelte Algoritmus zur Berechnung aller Wissensbasen über einer Signatur implementiert. Dabei soll ergründet werden, welchen Aufwand diese Generierung mit einer steigender Anzahl der darin  enthaltenden Konditionalen verursacht. Die als Ergebnis entstandenen Wissensbasen sollen dann dazu dienen, das System InfOCF (siehe \cite{beierle17}) experimentell zu evaluieren.
\\
Der weitere Ablauf dieser Arbeit ist wie folgt: Im nächsten Kapitel werden zuerst die Grundlagen zu Konditionalen und Wissensbasen dargestellt um die Basis für das weitere vorgehen zu schaffen. Anschließend werden die betrachteten Koditionalen in einer Normalform vollständig berechnet und geordnet gespeichert. Im Hauptteil dieser Arbeit wird dann beschrieben, wie aus diesen Kondititionalen mit einem Algoritmus \cite{beierle19} Wissensbasen erzeugt werden können. Die erhaltenen Ergebnisse werden dann in einer geeigneten Form dargestellt.
\section{Grundlagen zu den Konditionalen}
In diesem Abschnitt werden kurz die Grundlagen dargestellt, auf denen der Hauptteil dieser Arbeit basiert. Die Inhalte und Darstellungen sind dabei aus \cite{beierle19} übernommen.
\subsection{Darstellung}
Im Folgenden wird kurz die Notation beschrieben, mit der die Konditionale im weiteren Verlauf der Arbeit dargestellt werden. Einzelne Einheiten, die beschrieben werden, werden mit Kleinbuchstaben $a, b, c ...$ bezeichnet. Aussagen über diese werden mit den dazugehörigen Großbuchstaben $A, B, C, ...$ formuliert, die Gesamtheit der Sprache wird als $\lag$ bezeichnet. Die Menge der möglichen Welten, die mit $\lag$ beschrieben wird, wird mit $\Omega$ bezeichnet. In einer Welt $\omega \in \Omega$  hat der Ausdruck $\omega \models A$ die Bedeutung, dass $A \in \lag$ in dieser Welt gilt. Des weiteren wird $\neg A$ als $\overline{A}$ und $A \wedge B$ als $AB$ dargestellt. Diese Arbeit beschränkt sich dabei auf die Betrachtung an Aussagen, die über der Signatur $\sum_{abc} = \{a, b, c\}$ gebildet werden können. \\
Ein Konditional ist die Verbindung aus einer Vorbedingung und einer Folgerung aus der Vorbedingung. Ein qualitatives Konditional ist ein solches unter  Unsicherheit und entspricht einem Zusammenhang in der Form \glqq Wenn $A$, dann normalerweise $B$\grqq . Dargestellt wird so ein Konditional  im folgenden mit dem Symbol $|$ in der Form $r = ( \lag | \lag)$. Ein Beispiel ist $r = (B|A)$, das dazugehörige gegenteilige Konditional lautet dann $\overline{r} = (\overline{B}|A)$.\\
Als Semantik verwenden wir ordinale Konditionale nach Wolfgang Spohn, siehe dazu auch \cite{spohn12}. Eine solche Funktion ist eine Funktion in der Form $\kappa :  \Omega \rightarrow \mathbb{N} $. Hiermit wird einer Welt $\kappa$ aus den Möglichen Welten $\Omega$ eine natürliche Zahl zugeordnet, um den Grad an Plausibilität zu beschreiben. Die normalste aller Welten bekommt den Wert 0, unplausiblerere bekommen aufsteigend entsprechend ihrer Unplausibilität höhere Werte zugewiesen. Ein Konditional wird in diesem System akzeptiert, wenn es plausibler ist, also einen geringeren Rang besitzt, als dessen Negation, z.B. $\kappa(AB)<\kappa(A\overline{B})$. \\
Eine endliche Menge $\mathcal{R} \subseteq (\lag | \lag)$ an Konditionalen wird als Wissensbasis bezeichnet. Eine Wissensbasis wird als konsistent bezeichnet, wenn eine konditionale Rangfunktion existiert, die alle $\mathcal{R}$ akzeptiert.

\subsection{Äquivalenz und Normalform von Konditionalen}
Äquivalenz für Konditionale soll im weiteren Verlauf dadurch definiert sein, dass zwei äquivalente Konditionale die Menge möglicher Welten auf die gleiche Art und Weise aufteilen:
\begin{equation}
(B|A)\equiv (B^\prime|A^\prime) \quad iff \quad A\equiv A^\prime \quad and \quad AB \equiv A^\prime B^\prime
\end{equation}
Im folgenden wird eine Normalform für Konditionale beschrieben. Das Ziel ist, eine klare Definition einer vollständigen und minimalen  Menge an Konditionalen über einer Signatur $\sum$, die untereinander nicht äquivalent sind. Diese Normalform wird mit $NFC(\sum)$ bezeichnet.
\begin{theorem}
$NFC(\sum) = \{(B|A)|A \subseteq
 \Omega_A, B \subsetneq A, B \neq \emptyset \}$ ist die Normalform für Konditionale. In diesen gilt dann:
 Nichttrivialität: $NFC$ enthält kein triviales Konditional.
 Komplettheit: für jedes nichttriviale Konditional in $\sum$ existiert ein äquivalentes Konditional in $NFC$
 Minimal: alle Konditionale in $NFC(\sum)$ sind paarweise nicht äquivalent.
\\ !!! hier muss noch was rein zur Menge der Konditionale
\end{theorem}
\subsection{Ordnungsrelation für Konditionale}
In diesem Abschnitt wird eine Ordnungsrelation für Konditionale definiert, unter der sie später bearbeitet und gespeichert werden können. Mit $\leq_{lex}$ wird im folgenden eine Ordnungsrelation für Zeichen und mit  $\leq_{set}$ eine Ordnungsrelation für Mengen bezeichnet.
\begin{theorem}[Ordnungsrelation für Zeichen und Mengen]
\begin{equation}\leq_{lex} \text{entspricht der normalen alphabetischen Ordnung der Buchstaben \ } a, b, c ...
\end{equation}
Für geordnete Mengen $S, S^\prime$ mit $S = \{e_1, ..., e_n\}$ und $S^\prime = \{e_{1}, ... , e_{n\prime}\}$
\begin{equation}
 S \leq_{set} S ^\prime \quad iff \quad n<n^\prime, \text{ oder } n = n ^\prime \text{ und } e_1...e_n  \leq_{lex}  e_1^\prime ... e_{n^\prime}^\prime
\end{equation}
\end{theorem}
!!hier jetzt ein beispiel. mit a,b.c??
\subsection{Wissensbasen}
Was ist das, Konsistenz von Wissensbasen mit Beispiel, Äquivalenzen? und Isomorphismen?
\section{Generierung der Wissensbasen}
\subsection{Vorüberlegungen}
\subsection{Berechnung der Konditionale}
\subsection{Der Algorithmus}
\subsection{Generierung von Wissensbasen}
\section{Bewertung der Ergebnisse}

\newpage
\bibliographystyle{alpha}
\bibliography{lit} 
\end{document}